\usepackage{geometry}
\usepackage{fontspec}
\usepackage{setspace}
\usepackage{titlesec}
\usepackage{titletoc}
\usepackage{array}
\usepackage{tabularx}
\usepackage{multirow}
\usepackage{fancyhdr}
\usepackage{zhnumber}
\usepackage{graphicx}
\usepackage{amsmath}
\usepackage{amssymb}
\usepackage{booktabs}
\usepackage{float}
\usepackage{listings}
\usepackage{xcolor}
\usepackage{subcaption} 
\usepackage[hidelinks]{hyperref}
\usepackage{graphicx}
\graphicspath{{data/}}
\geometry{left=3.17cm, right=3.17cm, top=2.54cm, bottom=2.54cm}
\onehalfspacing
\setmainfont{Times New Roman}

\lstset{
	basicstyle=\footnotesize\ttfamily,
	keywordstyle=\color{blue}\bfseries,
	commentstyle=\color{green!50!black},
	stringstyle=\color{red},
	frame=single,
	numbers=left,
	numberstyle=\tiny\color{gray},
	breaklines=true,
	captionpos=b,
	tabsize=4,
	language=Python
}

\ctexset{
	chapter = {
		name = {第,章},
		number = \chinese{chapter},
		format = \centering\heiti\zihao{-3}\bfseries,
		aftername = \quad,
		beforeskip = 10pt,
		afterskip = 20pt,
	},
	section = {
		format = \heiti\zihao{4}\bfseries,
		aftername = \quad,
	},
	subsection = {
		format = \heiti\zihao{4}\bfseries,
		aftername = \quad,
		indent = 0pt,
	},
	contentsname={目\quad\quad 录}
}

\titlecontents{chapter}[0pt]{\addvspace{6pt}\bfseries}{\thecontentslabel\quad}{}{\titlerule*[0.5pc]{.}\contentspage}

% ==========================================
% 6. 摘要环境定义 (重构版 - 仿照封面风格)
% ==========================================
\makeatletter

% 1. 定义内部存储变量
\newcommand{\@abstr@cn}{}   % 中文摘要内容
\newcommand{\@abstr@cnkw}{} % 中文关键词
\newcommand{\@abstr@en}{}   % 英文摘要内容
\newcommand{\@abstr@enkw}{} % 英文关键词

% 2. 定义统一的 abstracts 环境
\newenvironment{abstracts}{%
    % --- 环境开始:定义命令捕获用户输入 ---
    \newcommand{\cnabstract}[1]{\renewcommand{\@abstr@cn}{##1}}
    \newcommand{\cnkeywords}[1]{\renewcommand{\@abstr@cnkw}{##1}}
    \newcommand{\enabstract}[1]{\renewcommand{\@abstr@en}{##1}}
    \newcommand{\enkeywords}[1]{\renewcommand{\@abstr@enkw}{##1}}
}{%
    % --- 环境结束:执行排版输出 ---
    
    % >>> 输出中文摘要页
    \clearpage
    \phantomsection
    \addcontentsline{toc}{chapter}{摘要}
    \markboth{摘要}{摘要}
    
    \begin{center}
        \heiti\zihao{3}\bfseries 摘\quad\quad 要
    \end{center}
    
    \vspace{1em} 
    \fangsong\zihao{-4}
    \@abstr@cn  % 输出中文内容
    \par 
    \vspace{1em}
    \noindent\textbf{\heiti 关键词:}\@abstr@cnkw % 输出中文关键词
    
    % >>> 输出英文摘要页
    \clearpage
    \phantomsection
    \addcontentsline{toc}{chapter}{ABSTRACT}
    \markboth{ABSTRACT}{ABSTRACT}
    
    \begin{center}
        \zihao{3}\bfseries ABSTRACT
    \end{center}
    
    \vspace{1em} 
    \zihao{-4}
    \@abstr@en  % 输出英文内容
    \par 
    \vspace{1em}
    \noindent\textbf{Key words: }\@abstr@enkw % 输出英文关键词
}
\makeatother

% ==========================================
% 5. 封面环境定义 (新增)
% ==========================================
\makeatletter

% 定义内部存储变量,默认留空
\newcommand{\@cover@title}{}
\newcommand{\@cover@department}{}
\newcommand{\@cover@major}{}
\newcommand{\@cover@name}{}
\newcommand{\@cover@stuid}{}
\newcommand{\@cover@instructor}{}
\newcommand{\@cover@grade}{}

% 定义 cover 环境
\newenvironment{cover}{%
    % --- 环境开始:重定义命令以捕获用户输入 ---
    \renewcommand{\title}[1]{\renewcommand{\@cover@title}{##1}}
    \newcommand{\department}[1]{\renewcommand{\@cover@department}{##1}}
    \newcommand{\major}[1]{\renewcommand{\@cover@major}{##1}}
    \newcommand{\name}[1]{\renewcommand{\@cover@name}{##1}}
    \newcommand{\stuid}[1]{\renewcommand{\@cover@stuid}{##1}}
    \newcommand{\instructor}[1]{\renewcommand{\@cover@instructor}{##1}}
    \newcommand{\grade}[1]{\renewcommand{\@cover@grade}{##1}}
}{%
    % --- 环境结束:执行实际的排版逻辑 ---
    \begin{titlepage}
        \centering
        \vspace*{2cm}
        
        % 标题部分
        {\heiti\zihao{1} 课程设计报告 \par}
        
        \vspace{4cm}
        
        % 表格部分
        {\zihao{3} \songti
            \begin{tabular}{rc}
                题\quad\quad 目 & \underline{\makebox[9cm]{\@cover@title}}      \\[1.5cm]
                学\quad\quad 院 & \underline{\makebox[9cm]{\@cover@department}} \\[0.8cm]
                专\quad\quad 业 & \underline{\makebox[9cm]{\@cover@major}}      \\[0.8cm]
                姓\quad\quad 名 & \underline{\makebox[9cm]{\@cover@name}}       \\[0.8cm]
                学\quad\quad 号 & \underline{\makebox[9cm]{\@cover@stuid}}      \\[0.8cm]
                任课教师        & \underline{\makebox[9cm]{\@cover@instructor}} \\[0.8cm]
                成\quad\quad 绩 & \underline{\makebox[9cm]{\@cover@grade}}      \\
            \end{tabular}
        }
        
        \vfill
        
        % 日期部分
        {\zihao{3} \the\year \quad 年 \quad \the\month \quad 月 \quad \the\day \quad 日}
        \vspace{2cm}
    \end{titlepage}
}
\makeatother