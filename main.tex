\documentclass[a4paper,zihao=-4]{ctexrep}
\usepackage{geometry}
\usepackage{fontspec}
\usepackage{setspace}
\usepackage{titlesec}
\usepackage{titletoc}
\usepackage{array}
\usepackage{tabularx}
\usepackage{multirow}
\usepackage{fancyhdr}
\usepackage{zhnumber}
\usepackage{graphicx}
\usepackage{amsmath}
\usepackage{amssymb}
\usepackage{booktabs}
\usepackage{float}
\usepackage{listings}
\usepackage{xcolor}
\usepackage{subcaption} 
\usepackage[hidelinks]{hyperref}
\usepackage{graphicx}
\graphicspath{{data/}}
\geometry{left=3.17cm, right=3.17cm, top=2.54cm, bottom=2.54cm}
\onehalfspacing
\setmainfont{Times New Roman}

\lstset{
	basicstyle=\footnotesize\ttfamily,
	keywordstyle=\color{blue}\bfseries,
	commentstyle=\color{green!50!black},
	stringstyle=\color{red},
	frame=single,
	numbers=left,
	numberstyle=\tiny\color{gray},
	breaklines=true,
	captionpos=b,
	tabsize=4,
	language=Python
}

\ctexset{
	chapter = {
		name = {第,章},
		number = \chinese{chapter},
		format = \centering\heiti\zihao{-3}\bfseries,
		aftername = \quad,
		beforeskip = 10pt,
		afterskip = 20pt,
	},
	section = {
		format = \heiti\zihao{4}\bfseries,
		aftername = \quad,
	},
	subsection = {
		format = \heiti\zihao{4}\bfseries,
		aftername = \quad,
		indent = 0pt,
	},
	contentsname={目\quad\quad 录}
}

\titlecontents{chapter}[0pt]{\addvspace{6pt}\bfseries}{\thecontentslabel\quad}{}{\titlerule*[0.5pc]{.}\contentspage}

% ==========================================
% 6. 摘要环境定义 (重构版 - 仿照封面风格)
% ==========================================
\makeatletter

% 1. 定义内部存储变量
\newcommand{\@abstr@cn}{}   % 中文摘要内容
\newcommand{\@abstr@cnkw}{} % 中文关键词
\newcommand{\@abstr@en}{}   % 英文摘要内容
\newcommand{\@abstr@enkw}{} % 英文关键词

% 2. 定义统一的 abstracts 环境
\newenvironment{abstracts}{%
    % --- 环境开始:定义命令捕获用户输入 ---
    \newcommand{\cnabstract}[1]{\renewcommand{\@abstr@cn}{##1}}
    \newcommand{\cnkeywords}[1]{\renewcommand{\@abstr@cnkw}{##1}}
    \newcommand{\enabstract}[1]{\renewcommand{\@abstr@en}{##1}}
    \newcommand{\enkeywords}[1]{\renewcommand{\@abstr@enkw}{##1}}
}{%
    % --- 环境结束:执行排版输出 ---
    
    % >>> 输出中文摘要页
    \clearpage
    \phantomsection
    \addcontentsline{toc}{chapter}{摘要}
    \markboth{摘要}{摘要}
    
    \begin{center}
        \heiti\zihao{3}\bfseries 摘\quad\quad 要
    \end{center}
    
    \vspace{1em} 
    \fangsong\zihao{-4}
    \@abstr@cn  % 输出中文内容
    \par 
    \vspace{1em}
    \noindent\textbf{\heiti 关键词:}\@abstr@cnkw % 输出中文关键词
    
    % >>> 输出英文摘要页
    \clearpage
    \phantomsection
    \addcontentsline{toc}{chapter}{ABSTRACT}
    \markboth{ABSTRACT}{ABSTRACT}
    
    \begin{center}
        \zihao{3}\bfseries ABSTRACT
    \end{center}
    
    \vspace{1em} 
    \zihao{-4}
    \@abstr@en  % 输出英文内容
    \par 
    \vspace{1em}
    \noindent\textbf{Key words: }\@abstr@enkw % 输出英文关键词
}
\makeatother

% ==========================================
% 5. 封面环境定义 (新增)
% ==========================================
\makeatletter

% 定义内部存储变量,默认留空
\newcommand{\@cover@title}{}
\newcommand{\@cover@department}{}
\newcommand{\@cover@major}{}
\newcommand{\@cover@name}{}
\newcommand{\@cover@stuid}{}
\newcommand{\@cover@instructor}{}
\newcommand{\@cover@grade}{}

% 定义 cover 环境
\newenvironment{cover}{%
    % --- 环境开始:重定义命令以捕获用户输入 ---
    \renewcommand{\title}[1]{\renewcommand{\@cover@title}{##1}}
    \newcommand{\department}[1]{\renewcommand{\@cover@department}{##1}}
    \newcommand{\major}[1]{\renewcommand{\@cover@major}{##1}}
    \newcommand{\name}[1]{\renewcommand{\@cover@name}{##1}}
    \newcommand{\stuid}[1]{\renewcommand{\@cover@stuid}{##1}}
    \newcommand{\instructor}[1]{\renewcommand{\@cover@instructor}{##1}}
    \newcommand{\grade}[1]{\renewcommand{\@cover@grade}{##1}}
}{%
    % --- 环境结束:执行实际的排版逻辑 ---
    \begin{titlepage}
        \centering
        \vspace*{2cm}
        
        % 标题部分
        {\heiti\zihao{1} 课程设计报告 \par}
        
        \vspace{4cm}
        
        % 表格部分
        {\zihao{3} \songti
            \begin{tabular}{rc}
                题\quad\quad 目 & \underline{\makebox[9cm]{\@cover@title}}      \\[1.5cm]
                学\quad\quad 院 & \underline{\makebox[9cm]{\@cover@department}} \\[0.8cm]
                专\quad\quad 业 & \underline{\makebox[9cm]{\@cover@major}}      \\[0.8cm]
                姓\quad\quad 名 & \underline{\makebox[9cm]{\@cover@name}}       \\[0.8cm]
                学\quad\quad 号 & \underline{\makebox[9cm]{\@cover@stuid}}      \\[0.8cm]
                任课教师        & \underline{\makebox[9cm]{\@cover@instructor}} \\[0.8cm]
                成\quad\quad 绩 & \underline{\makebox[9cm]{\@cover@grade}}      \\
            \end{tabular}
        }
        
        \vfill
        
        % 日期部分
        {\zihao{3} \the\year \quad 年 \quad \the\month \quad 月 \quad \the\day \quad 日}
        \vspace{2cm}
    \end{titlepage}
}
\makeatother
\begin{document}
	\begin{cover}
        \title{报告适用 LaTeX 模板}
        \department{数学与人工智能}
        \major{人工智能}
        \name{qwedc001}
        \stuid{998244353}
        \instructor{Gemini 3 Pro}
        % \grade{} 
    \end{cover}
    
	\tableofcontents
    
    \begin{abstracts}
        \cnabstract{
            为了摆脱用 Word 写各种复杂的机器学习公式,我用 Gemini 3 Pro 喂出了一个和我们 Word 模板差不多样式的 \LaTeX 文件,然后顺利的用 \LaTeX 完成了工作。

            所以我就在想,把这个文件模板化并且开源,是不是就能造福一些也会写 \LaTeX 并且不想被 Word 模板折磨的朋友。

            所以我就这么干了。
        }
        \cnkeywords{\LaTeX 模板;AI 真是太好用了}
        
        \enabstract{
            To avoid using Word to write those shitty formulas,I've used Gemini 3 (Thanks Google) to generate a \LaTeX file that looks similar to the original Word template.And I've finished my work.

            Got me thinkin',I should make it opensource.

            So I did this.
        }
        \enkeywords{\LaTeX \enspace template;Thanks Google}
    
    \end{abstracts}
	
	\chapter{绪论}
	\songti
	\section{研究背景}
	就如摘要中所说,我本人本来是要自己做一个机器学习项目,但是被项目所需要的各种各样的代码,公式,图表插入搞的根本不想用学院提供的 Word 模板。要命的是学院只有 Word 模板。
	
	\section{本文主要工作}
	针对上述问题,本文提出了一套解决方案,主要贡献如下:
	\begin{itemize}
		\item 允许使用者在 main.tex 里撰写整个报告内容,并且不用担心排版问题。
		\item 重定义默认存放图片数据的根目录为 data 目录。例如你想引用 data/img1.png,只需要写以下内容。
        \begin{lstlisting}[language=TeX, caption=引用图片举例]
\includegraphics{img1.png}
	    \end{lstlisting}
		\item 允许通过 appendix 目录实现自定义引用内容:我在里面放了我们学校用的教师评价表。
	\end{itemize}
	
	%---------------------------------------------------------------------------
	% 第二章 核心算法设计
	%---------------------------------------------------------------------------
	\chapter{核心算法与系统设计}
	
	\section{预定义:setup.tex}
	相比于直接把宏定义堆在 main.tex 中,我选择拆分出 setup.tex 来让 main 看上去更清爽一点。
    \section{预定义:appendix}
	你可以在这里添加自己想要的自定义界面。
    
    
	\chapter{实验环境与工程实现}
	
	\section{硬件与软件环境}
	本模板使用了多个软硬件环境进行编写,具体配置如下:
	\begin{table}[H]
		\centering
		\caption{编写环境配置}
		\begin{tabular}{ll}
			\toprule
			项目 & 配置详情 \\
			\midrule
			\textbf{Overleaf} & overleaf.com \\
			\textbf{Gemini} & gemini.google.com \\
			\textbf{我自己的电脑} & HP Omen 9 Pro \\
			\bottomrule
		\end{tabular}
	\end{table}
	
	
	%---------------------------------------------------------------------------
	% 第四章 实验结果分析
	%---------------------------------------------------------------------------
	\chapter{实验结果与分析}
	
	\section{模板效果分析}
	有效的帮我完成了我的报告任务。
	
	\section{模板作用评估}
	\subsection{定量指标}
	开源到 Github 上收集更多数据以后再更新。
	
	\begin{table}[H]
		\centering
		\caption{写作方法对比}
		\begin{tabular}{lc}
			\toprule
			写作方法 & 痛度 \\
			\midrule
			Word & Severe \\
			TexStudio + 单文件 report.tex & Mild \\
			\textbf{Overleaf + 多文件模板} & \textbf{No pain} \\
			\bottomrule
		\end{tabular}
	\end{table}
	
	\textbf{结果解读}:
	\begin{enumerate}
		\item 我认为模板和 Overleaf 很对。
	\end{enumerate}
	
	\chapter{总结与展望}
	希望本模板能提高各位写作效率。

	\clearpage
	\phantomsection
	\addcontentsline{toc}{chapter}{参考文献}
	\renewcommand{\bibname}{\heiti\zihao{4}\bfseries 参考文献}
	
	\begin{thebibliography}{99}
		\zihao{5}
		\setlength{\itemsep}{0pt}
		\bibitem{ref1} Qilu University of Technology teachers Report Template//Class idk
	\end{thebibliography}
	
	\clearpage
\thispagestyle{empty}

\begin{center}
    {\heiti\zihao{1} 课程设计验收评语表}
\end{center}
\vspace{0.5cm}

\begin{table}[h!]
    \centering
    \renewcommand{\arraystretch}{1.2}
    % 注意:X 列是自动换行的,所以内部换行要小心
    \begin{tabularx}{\textwidth}{|X|}
        \hline
        % --- 第一栏:教师评语 ---
        \vspace{0.5em}
        {\heiti\zihao{4} \textbf{一、 教师评语}} \\ % 这里换行是在单元格内部换行
        \rule{0pt}{10cm} % 撑开高度
        \hfill {\songti\zihao{4} 教师签名:\underline{\hspace{5cm}}} \quad \quad \\[1em]
        \hfill {\songti\zihao{4} \quad 年 \quad 月 \quad 日} \quad \quad \\[0.5em]
        \hline
        
        % --- 第二栏:最终成绩 ---
        \vspace{0.5em}
        {\heiti\zihao{4} \textbf{二、 最终成绩}} \par % 使用 \par 结束标题段落
        \vspace{1em}
        
        % 【修复重点】使用花括号限制 \centering 的作用范围
        {
            \centering 
            \rule{0pt}{2cm} % 撑开高度
            {\heiti\zihao{1} \underline{\hspace{6cm}}} % 成绩下划线
            \par % 结束居中段落
        }
        
        \vspace{1em}
        \\ % 这里的 \\ 现在可以正常结束表格行了
        \hline
        
        % --- 第三栏:备注 ---
        \vspace{0.5em}
        {\heiti\zihao{4} \textbf{三、 备注}} \\
        \rule{0pt}{4cm} % 撑开高度
        \\
        \hline
    \end{tabularx}
\end{table}
	
\end{document}